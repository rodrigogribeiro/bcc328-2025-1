% Intended LaTeX compiler: pdflatex
\documentclass[11pt]{article}
\usepackage[utf8]{inputenc}
\usepackage[T1]{fontenc}
\usepackage{graphicx}
\usepackage{longtable}
\usepackage{wrapfig}
\usepackage{rotating}
\usepackage[normalem]{ulem}
\usepackage{amsmath}
\usepackage{amssymb}
\usepackage{capt-of}
\usepackage{hyperref}
\author{Construção de compiladores I}
\date{}
\title{Um Compilador Simples}
\hypersetup{
 pdfauthor={Construção de compiladores I},
 pdftitle={Um Compilador Simples},
 pdfkeywords={},
 pdfsubject={},
 pdfcreator={Emacs 29.4 (Org mode 9.7.22)}, 
 pdflang={English}}
\begin{document}

\maketitle
\section*{Objetivos}
\label{sec:orgf9eeae1}

\subsection*{Objetivos}
\label{sec:org844caa2}

\begin{itemize}
\item Apresentar a especificação léxica e sintática de uma linguagem simples.

\item Apresentar a implementação completa de um compilador para uma linguagem simples de expressões.
\end{itemize}
\section*{Expressões Aritméticas}
\label{sec:org9c50925}

\subsection*{Expressões Aritméticas}
\label{sec:orgccb6787}

\begin{itemize}
\item Especificação léxica
\begin{itemize}
\item Números: digit+
\item Operadores e separadores: +, *, (, )
\end{itemize}
\end{itemize}
\subsection*{Expressões Aritméticas}
\label{sec:orgf25e423}

\begin{itemize}
\item Especificação sintática
\end{itemize}

\begin{array}{lcl}
   e & \to & n \,|\, e + e \,|\, e * e \,|\, (e)
\end{array}
\subsection*{Expressões Aritméticas}
\label{sec:org44fcfae}

\begin{itemize}
\item De posse da gramática, vamos considerar as diferentes etapas do compilador. 
\begin{itemize}
\item Análise léxica
\item Análise sintática
\item Intepretador
\item Geração de código
\end{itemize}
\end{itemize}
\section*{Análise léxica}
\label{sec:org6a83778}

\subsection*{Análise léxica}
\label{sec:org93b8168}

\begin{itemize}
\item Primeira etapa do front-end de um compilador.
\item Simplificar a entrada para análise sintática.
\end{itemize}
\subsection*{Análise léxica}
\label{sec:orgc477053}

\begin{itemize}
\item Simplificações:
\begin{itemize}
\item Remoção de espaços em branco.
\item Remoção de comentários.
\end{itemize}

\item Resultado: lista de \textbf{\textbf{tokens}}.
\end{itemize}
\subsection*{Análise léxica}
\label{sec:orgf24635c}

\begin{itemize}
\item Token
\begin{itemize}
\item Componente indivisível da sintaxe de uma linguagem.
\end{itemize}
\item Exemplos:
\begin{itemize}
\item identificadores
\item palavras reservadas
\item separadores
\item literais
\end{itemize}
\end{itemize}
\subsection*{Análise léxica}
\label{sec:orga1ccf91}

\begin{itemize}
\item Como implementar a análise léxica?
\end{itemize}
\subsection*{Análise léxica ad-hoc}
\label{sec:orge42cf5c}

\begin{itemize}
\item Percorra a string:
\begin{itemize}
\item Se for um dígito, guarde-o para formar um número.
\item Se for um operador, gere o token.
\item Se for um parêntesis, gere o token.
\item Se for um espaço, tente gerar um número e descarte o espaço.
\end{itemize}
\end{itemize}
\subsection*{Análise léxica ad-hoc}
\label{sec:orgb625dec}

\begin{itemize}
\item Como representar tokens?
\begin{itemize}
\item Primeiro, definimos seus tipos: lexemas.
\end{itemize}
\end{itemize}

\begin{verbatim}
data Lexeme  
  = TNumber Value  
  | TPlus 
  | TMul
  | TLParen
  | TRParen 
\end{verbatim}
\subsection*{Análise léxica ad-hoc}
\label{sec:org3fed344}

\begin{itemize}
\item Token = lexema + posição
\end{itemize}

\begin{verbatim}
data Token 
  = Token {
      lexeme :: Lexeme
    , position :: (Int, Int)
    } 
\end{verbatim}
\subsection*{Análise léxica ad-hoc}
\label{sec:orged9d1fd}

\begin{itemize}
\item Configuração do analisador léxico
\begin{itemize}
\item Linha e coluna atual.
\item String de dígitos consecutivos encontrados.
\item Lista de tokens encontrados.
\end{itemize}
\end{itemize}

\begin{verbatim}
type Line = Int 
type Column = Int
type State = (Line, Column, String, [Token])
\end{verbatim}
\subsection*{Análise léxica ad-hoc}
\label{sec:orgb195486}

\begin{itemize}
\item Transição de estado sob um caractere
\end{itemize}

\begin{verbatim}
transition :: State -> Char -> Either String State
transition state@(l, col, t, ts) c 
  | c == '\n' = mkDigits state c 
  | isSpace c = mkDigits state c 
  | c == '+' = Right ( l, col + 1, ""
                     , mkToken state (Token TPlus (l,col)) ++ ts)
  | isDigit c = Right (l, col + 1, c : t, ts)
  | otherwise = unexpectedCharError l col c 
\end{verbatim}
\subsection*{Análise léxica ad-hoc}
\label{sec:org18c2d63}

\begin{itemize}
\item Criando token de dígitos.
\end{itemize}

\begin{verbatim}
mkDigits :: State -> Char -> Either String State  
mkDigits state@(l, col, s, ts) c 
  | null s = Right state 
  | all isDigit s 
      = let t = Token (TNumber (VInt (read $ reverse s))) (l,col)
            l' = if c == '\n' then l + 1 else l 
            col' = if c /= '\n' && isSpace c then col + 1 else col  
        in Right (l', col', "", t : ts)
  | otherwise = unexpectedCharError l col c
\end{verbatim}
\subsection*{Análise léxica ad-hoc}
\label{sec:org5dca865}

\begin{itemize}
\item Criando tokens
\end{itemize}

\begin{verbatim}
mkToken :: State -> Token -> [Token]
mkToken (l,c, s@(_ : _), _) t 
  | all isDigit s = [t, Token (TNumber (VInt (read $ reverse s))) (l,c)]
  | otherwise = [t]
mkToken _ t = [t] 
\end{verbatim}
\subsection*{Analise léxica ad-hoc}
\label{sec:org4822d3d}

\begin{itemize}
\item Analisador léxico 
\begin{itemize}
\item Caminhamento na entrada feito pela função foldl
\end{itemize}
\end{itemize}

\begin{verbatim}
lexer :: String -> Either String [Token]
lexer = either Left (Right . extract) . foldl step start 
  where 
    start = Right (1,1,"",[])
    step ac@(Left _) _ = ac 
    step (Right state) c = transition state c  

    extract (l, col, s, ts) 
      | null s = reverse ts 
      | otherwise 
        = let t = Token (TNumber (VInt (read $ reverse s))) (l, col)
          in reverse (t : ts)
\end{verbatim}
\section*{Análise sintática}
\label{sec:org2416232}

\subsection*{Análise sintática}
\label{sec:org824b457}

\begin{itemize}
\item Vamos considerar uma técnica de análise sintática chamada de análise descendente recursiva.
\item Permite a construção manual de analisadores sintáticos.
\end{itemize}
\subsection*{Análise sintática}
\label{sec:org2de1ce6}

\begin{itemize}
\item Analisador descendente recursivo 
\begin{itemize}
\item Uma função para cada não terminal da gramática
\end{itemize}
\end{itemize}
\subsection*{Análise sintática}
\label{sec:org3a3a118}

\begin{itemize}
\item Lados direitos de regra como corpo da função 
\begin{itemize}
\item Caso um elemento de regra seja um token, consumimos este token
\item Caso seja um não terminal, chamamos a função correspondente.
\end{itemize}
\end{itemize}
\subsection*{Análise sintática}
\label{sec:org6582ee3}

\begin{itemize}
\item Analisadores descentes recursivos não podem ser implementados para gramáticas recursivas à esquerda.
\begin{itemize}
\item Gramáticas recursivas à esquerda geram parsers que entram em loop infinito!
\end{itemize}
\end{itemize}
\subsection*{Análise sintática}
\label{sec:org2296ca1}

\begin{itemize}
\item Gramática para expressões
\begin{itemize}
\item Regras recursivas à esquerda.
\end{itemize}

\begin{array}{lcl}
  E & \to & n \,|\, E + E\,|\, E * E\,|\,(E)
\end{array}
\end{itemize}
\subsection*{Análise sintática}
\label{sec:org46343a7}

\begin{itemize}
\item Gramática para expressões 
\begin{itemize}
\item Sem regras recursivas à esquerda.
\end{itemize}

\begin{array}{lcl}
  E & \to & T + E\,|\,T\\
  T & \to & F * T\,|\,F\\
  F & \to & n \,|\, (E)\\
\end{array}
\end{itemize}
\subsection*{Análise sintática}
\label{sec:org3348bbd}

\begin{itemize}
\item De posse de uma gramática adequada, podemos partir para uma implementação.

\item Em linguagens funcionais, analisadores descendentes recursivos são implementados como combinadores.
\end{itemize}
\subsection*{Análise sintática}
\label{sec:org904a39e}

\begin{itemize}
\item Um parser é uma função:
\begin{itemize}
\item Entrada uma lista de tokens (tipo s)
\end{itemize}
\end{itemize}
\subsection*{Análise sintática}
\label{sec:org005f3f3}
\begin{itemize}
\item Resultado: uma lista de pares de resultados e o restante de tokens. 
\begin{itemize}
\item Lista vazia: erro
\item Lista unitária: resultado único
\item Lista com mais de um resultado: possibilidade de backtracking.
\end{itemize}
\end{itemize}
\subsection*{Análise sintática}
\label{sec:orgbad0d19}

\begin{itemize}
\item Representando em Haskell
\end{itemize}

\begin{verbatim}
newtype Parser s a 
  = Parser { runParser :: [s] -> [(a, [s])] }
\end{verbatim}
\subsection*{Análise sintática}
\label{sec:org77bbf64}

\begin{itemize}
\item Processando um token.
\end{itemize}

\begin{verbatim}
sat :: (s -> Bool) -> Parser s s
sat p = Parser (\ ts -> case ts of 
                          [] -> [] 
                          (t' : ts') -> 
                            if p t' then [(t', ts')] 
                                    else [])
\end{verbatim}
\subsection*{Análise sintática}
\label{sec:org752aaff}

\begin{itemize}
\item Processando uma sequência de tokens.
\end{itemize}

\begin{verbatim}
token :: Eq s => [s] -> Parser s [s]
token s = Parser (\ ts -> if (take (length s) ts) == s 
                          then [(s, drop (length s) ts)]
                          else [])
\end{verbatim}
\subsection*{Análise sintática}
\label{sec:org32af6aa}

\begin{itemize}
\item Construção de resultados (instância de Functor)
\end{itemize}

\begin{verbatim}
instance Functor (Parser s) where 
  fmap f (Parser p) = Parser g 
    where g ts = [(f x, ts') | (x, ts') <- p ts]
\end{verbatim}
\subsection*{Análise sintática}
\label{sec:orgdb57a0e}

\begin{itemize}
\item Processando um dígito
\end{itemize}

\begin{verbatim}
digit :: Parser Char Char 
digit = sat isDigit
\end{verbatim}
\subsection*{Análise sintática}
\label{sec:org470cdb3}

\begin{itemize}
\item Concatenação de parser (instance de Applicative)
\end{itemize}

\begin{verbatim}
instance Applicative (Parser s) where 
  pure x = Parser (\ ts -> [(x,ts)])
  (Parser pf) <*> (Parser px)
    = Parser (\ ts -> [(f x, ts') | (f, ts1) <- pf ts
                                  , (x, ts') <- px ts1])
\end{verbatim}
\subsection*{Análise sintática}
\label{sec:org30a67f3}

\begin{itemize}
\item Alternativas de parsers (instance de Alternative)
\end{itemize}

\begin{verbatim}
instance Alternative (Parser s) where 
  empty = Parser (\ _ -> [])
  (Parser p1) <|> (Parser p2) = Parser f 
    where f ts = p1 ts ++ p2 ts
\end{verbatim}
\subsection*{Análise sintática}
\label{sec:orgea93aa4}

\begin{itemize}
\item Repetindo um parser
\end{itemize}

\begin{verbatim}
many :: Parser s a -> Parser s [a]
many p = (:) <$> p <*> many p <|> pure []
\end{verbatim}
\subsection*{Análise sintática}
\label{sec:org6eda6cf}

\begin{itemize}
\item Parser para números naturais
\end{itemize}

\begin{verbatim}
natural :: Parser Char Int 
natural 
  = foldl f 0 <$> many digit  
    where 
      f a b = a * 10 + b 
\end{verbatim}
\subsection*{Análise sintática}
\label{sec:org3882776}

\begin{itemize}
\item Executando um parser opcionalmente.
\end{itemize}

\begin{verbatim}
option :: Parser s a -> a -> Parser s a 
option p v = p <|> pure v 
\end{verbatim}
\subsection*{Análise sintática}
\label{sec:org424d82b}

\begin{itemize}
\item Parser para números inteiros
\end{itemize}

\begin{verbatim}
integer :: Parser Char Int 
integer = option (const negate <$> token "-") id <*> natural
\end{verbatim}
\subsection*{Análise sintática}
\label{sec:org7514f54}

\begin{itemize}
\item Lidando com separadores. 
\begin{itemize}
\item Separadores sem semântica
\end{itemize}
\end{itemize}

\begin{verbatim}
endBy :: Parser s a -> Parser s b -> Parser s [a]
p `endBy` e 
  = many (f <$> p <*> e) 
    where 
      f x _ = x 
\end{verbatim}
\subsection*{Análise sintática}
\label{sec:org5d2e9aa}

\begin{itemize}
\item Lidando com separadores 
\begin{itemize}
\item Separadores com semântica (operadores)
\end{itemize}
\end{itemize}

\begin{verbatim}
chainl :: Parser s (a -> a -> a) -> Parser s a -> Parser s a
chainl op p 
  = applyAll <$> p <*> many (flip <$> op <*> p)
    where 
      applyAll x [] = x
      applyAll x (f : fs) = applyAll (f x) fs
\end{verbatim}
\subsection*{Análise sintática}
\label{sec:orge6e3fd8}

\begin{itemize}
\item De posse de todas essas funções, podemos construir o analisador sintático para a gramática.

\begin{array}{lcl}
  E & \to & T + E\,|\,T\\
  T & \to & F * T\,|\,F\\
  F & \to & n \,|\, (E)\\
\end{array}
\end{itemize}
\subsection*{Análise sintática}
\label{sec:org04985e6}

\begin{itemize}
\item Antes de construir um parser, precisamos definir o resultado 
\begin{itemize}
\item Árvore de sintaxe abstrata.
\end{itemize}
\item Valores
\end{itemize}

\begin{verbatim}
data Value 
  = VInt Int
\end{verbatim}
\subsection*{Análise sintática}
\label{sec:org43c5d0a}

\begin{itemize}
\item Programas completos: expressões envolvendo adição e multiplicação.
\end{itemize}

\begin{verbatim}
data L0 
  = LVal Value  
  | LAdd L0 L0 
  | LMul L0 L0 
\end{verbatim}
\subsection*{Análise sintática}
\label{sec:org933aa05}

\begin{itemize}
\item Parser de valores
\end{itemize}

\begin{verbatim}
valueParser :: Parser Value
valueParser = f <$> sat (\ t -> case lexeme t of 
                                  TNumber _ -> True 
                                  _ -> False)
      where 
        f (Token (TNumber n) _) = n 
\end{verbatim}
\subsection*{Análise sintática}
\label{sec:org60659d3}

\begin{itemize}
\item Parsing de parêntesis.
\end{itemize}

\begin{verbatim}
parens :: Parser a -> Parser a 
parens p 
  = f <$> lparen <*> p <*> rparen 
    where 
      f _ x _ = x 

lparen :: Parser Token 
lparen = sat (\ t -> lexeme t == TLParen)

rparen :: Parser Token 
rparen = sat (\ t -> lexeme t == TRParen)
\end{verbatim}
\subsection*{Análise sintática}
\label{sec:org087b57c}

\begin{itemize}
\item Parser de fatores
\end{itemize}

\begin{verbatim}
factorParser :: Parser L0
factorParser 
  = (LVal <$> valueParser) <|> parens expParser 
\end{verbatim}
\subsection*{Análise sintática}
\label{sec:org5cf0743}

\begin{itemize}
\item Parser de termos
\end{itemize}

\begin{verbatim}
termParser :: Parser L0 
termParser 
  = chainl pmult factorParser
    where  
      pmult = (const LMul) <$> sat (\ t -> lexeme t == TMul)
\end{verbatim}
\subsection*{Análise sintática}
\label{sec:org4dc19c2}

\begin{itemize}
\item Parser de expressões
\end{itemize}

\begin{verbatim}
expParser :: Parser L0 
expParser 
  = chainl padd termParser
    where 
      padd = (const LAdd) <$> sat (\ t -> lexeme t == TPlus)
\end{verbatim}
\subsection*{Análise sintática}
\label{sec:orgc339211}

\begin{itemize}
\item Função top-level do analisador sintático
\end{itemize}

\begin{verbatim}
l0Parser :: [Token] -> Either String L0 
l0Parser tks 
  = case runParser expParser tks of 
      ((t, []) : _) -> Right t 
      _             -> Left "Parse error on program."
\end{verbatim}
\section*{Intepretador}
\label{sec:orgdfe22cc}

\subsection*{Interpretador}
\label{sec:orgde9cec1}

\begin{itemize}
\item De posse de um analisador sintático, podemos agora:
\begin{itemize}
\item Executar o código (interpretador)
\item Gerar código (compilador)
\end{itemize}
\end{itemize}
\subsection*{Interpretador}
\label{sec:org7bc21e2}

\begin{itemize}
\item Intepretador:
\end{itemize}

\begin{verbatim}
eval :: L0 -> Either String Value 
eval (LVal v) = Right v 
eval (LAdd l1 l2) 
  = do 
       v1 <- eval l1 
       v2 <- eval l2 
       v1 .+. v2 
eval (LMul l1 l2) 
  = do 
       v1 <- eval l1
       v2 <- eval l2 
       v1 .*. v2
\end{verbatim}
\subsection*{Interpretador}
\label{sec:org9a0cae1}

\begin{itemize}
\item Operações sobre valores.
\end{itemize}

\begin{verbatim}
(.+.) :: Value -> Value -> Either String Value 
(VInt n1) .+. (VInt n2) = Right (VInt (n1 + n2))
e1 .+. e2 = Left $ unwords ["Type error on:", pretty e1, "+", pretty e2] 
\end{verbatim}
\section*{Geração de código}
\label{sec:org3764367}

\subsection*{Geração de código}
\label{sec:org15ab5cd}

\begin{itemize}
\item Vamos agora considerar o problema de gerar código. 
\begin{itemize}
\item Para uma máquina virtual
\item Executável, gerando código fonte C, e usar o gcc para produzir o executável.
\end{itemize}
\end{itemize}
\subsection*{Geração de código}
\label{sec:orgfae5f2c}

\begin{itemize}
\item Gerar o código C correspondente consiste em:
\begin{itemize}
\item Produzir o código com uma função main.
\item Corpo da função possui uma variável que recebe o resultado da expressão.
\item Imprime o valor da variável usando printf.
\end{itemize}
\end{itemize}
\subsection*{Geração de código}
\label{sec:orgb5f508a}

\begin{itemize}
\item Exemplo:
\begin{itemize}
\item Considere a expressão: (1 + 2) * 3
\end{itemize}
\end{itemize}

\begin{verbatim}
#include <stdio.h>
// code generated for expressions
int main () {
   int val = (1 + 2) * 3;
   printf("%d", val);
   putchar('\n');
   return 0;
}
\end{verbatim}
\subsection*{Geração de código}
\label{sec:org537b220}

\begin{itemize}
\item Como produzir esse código C?
\begin{itemize}
\item Vamos criar funções para produzir a expressão.
\item Usar um ``template'' do corpo do código C.
\end{itemize}
\end{itemize}
\subsection*{Geração de código}
\label{sec:org7f639f8}

\begin{itemize}
\item Como produzir o texto da expressão a partir da AST?
\begin{itemize}
\item Essa é a operação inversa da análise sintática
\item Normalmente conhecida como pretty-print
\end{itemize}
\end{itemize}
\subsection*{Geração de código}
\label{sec:orga041f70}

\begin{itemize}
\item Para isso, vamos utilizar uma biblioteca Haskell para facilitar essa tarefa.

\item Para construir o pretty-print, vamos seguir a estrutura da gramática. 
\begin{itemize}
\item Vantagem: evita parêntesis desnecessários.
\end{itemize}
\end{itemize}
\subsection*{Geração de código}
\label{sec:org39d3c3c}

\begin{itemize}
\item Gramática

\begin{array}{lcl}
  E & \to & T + E\,|\,T\\
  T & \to & F * T\,|\,F\\
  F & \to & n \,|\, (E)\\
\end{array}
\end{itemize}
\subsection*{Geração de código}
\label{sec:org899f6e1}

\begin{itemize}
\item Primeiro nível do pretty-print
\end{itemize}

\begin{verbatim}
pprAdd :: L0 -> Doc
pprAdd (LAdd e1 e2)
  = hsep [pprAdd e1, text "+", pprAdd e2]
pprAdd other = pprMul other 
\end{verbatim}
\subsection*{Geração de código}
\label{sec:org0aa2566}

\begin{itemize}
\item Segundo nível do pretty-print
\end{itemize}

\begin{verbatim}
pprMul :: L0 -> Doc 
pprMul (LMul e1 e2) 
  = hsep [pprMul e1, text "*", pprMul e2]
pprMul other = pprFact other 
\end{verbatim}
\subsection*{Geração de código}
\label{sec:orgac7dd1d}

\begin{itemize}
\item Último nível do pretty-print
\end{itemize}

\begin{verbatim}
pprFact :: L0 -> Doc
pprFact (LVal v) = ppr v 
pprFact other = parens (ppr other)
\end{verbatim}
\subsection*{Geração de código}
\label{sec:org44e0011}

\begin{itemize}
\item Gerando o corpo do código C.
\end{itemize}

\begin{verbatim}
cL0Codegen :: L0 -> String 
cL0Codegen e 
  = unlines $ [ "#include <stdio.h>"
              , "// code generated for expressions"
              , "int main () {" ] ++
              (map (nest 3) (generateBody e)) ++
              [ nest 3 $ "putchar('\\n');"
              , nest 3 "return 0;"
              , "}"
              ] 
    where
      nest n v = replicate n ' ' ++ v
\end{verbatim}
\section*{Máquina virtual}
\label{sec:org3b3e97d}

\subsection*{Máquina virtual}
\label{sec:org6266fb4}

\begin{itemize}
\item Agora vamos considerar a geração de código para uma máquina virtual simples, chamada de V0.
\end{itemize}
\subsection*{Máquina virtual}
\label{sec:orgc179b63}

\begin{itemize}
\item Instruções:
\begin{itemize}
\item Push(n): empilha um valor.
\item Add: desempilha dois valores e empilha a sua soma.
\item Mul: desempilha dois valores e empilha o seu produto.
\item Print: desempilha um valor e o imprime no console.
\item Halt: termina a execução.
\end{itemize}
\end{itemize}
\subsection*{Máquina virtual}
\label{sec:orgec367df}

\begin{itemize}
\item Execução de uma instrução, modifica a pilha da máquina.
\end{itemize}
\subsection*{Máquina virtual}
\label{sec:org6f3b4de}

\begin{itemize}
\item Executando uma instrução
\end{itemize}

\begin{verbatim}
step :: Instr -> Stack -> IO (Either String Stack)
step (Push v) stk = pure (Right (v : stk))
step Add (v1 : v2 : stk)
  = case v1 .+. v2 of 
      Left err -> pure (Left err)
      Right v3 -> pure (Right (v3 : stk))
step Print (v : stk)
  = do 
      print (pretty v)
      pure (Right stk) 
\end{verbatim}
\subsection*{Máquina virtual}
\label{sec:orgcbf07a1}

\begin{itemize}
\item Executando uma lista de instruções.
\end{itemize}

\begin{verbatim}
interp :: Code -> Stack -> IO () 
interp [] _ = pure ()
interp (c : cs) stk 
  = do 
      r <- step c stk
      case r of 
        Right stk' -> interp cs stk'
        Left err -> putStrLn err 
\end{verbatim}
\subsection*{Máquina virtual}
\label{sec:orgb936bea}

\begin{itemize}
\item Compilando uma expressão
\end{itemize}

\begin{verbatim}
codegen' :: L0 -> Code 
codegen' (LVal v) = [Push v]
codegen' (LAdd l0 l1) 
  = codegen' l0 ++ codegen' l1 ++ [Add]
codegen' (LMul l0 l1) 
  = codegen' l0 ++ codegen' l1 ++ [Mul]
\end{verbatim}
\subsection*{Máquina virtual}
\label{sec:orgd55511a}

\begin{itemize}
\item Compilando uma expressão
\end{itemize}

\begin{verbatim}
v0Codegen :: L0 -> Code 
v0Codegen e = codegen' e ++ [Print, Halt]
\end{verbatim}
\section*{Conclusão}
\label{sec:org607c88f}

\subsection*{Conclusão}
\label{sec:org5983d65}

\begin{itemize}
\item Nesta aula, apresentamos uma implementação para expressões de:
\begin{itemize}
\item Intepretador.
\item Compilador, usando o GCC
\item Compilador para uma máquina virtual.
\end{itemize}
\end{itemize}
\section*{Tarefa}
\label{sec:org7fb5c4a}

\subsection*{Tarefa}
\label{sec:org20a67d0}

\begin{itemize}
\item Primeira tarefa: obter o ambiente de execução e realizar testes com o código de exemplo.
\end{itemize}
\end{document}
