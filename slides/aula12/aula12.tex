% Intended LaTeX compiler: pdflatex
\documentclass[11pt]{article}
\usepackage[utf8]{inputenc}
\usepackage[T1]{fontenc}
\usepackage{graphicx}
\usepackage{longtable}
\usepackage{wrapfig}
\usepackage{rotating}
\usepackage[normalem]{ulem}
\usepackage{amsmath}
\usepackage{amssymb}
\usepackage{capt-of}
\usepackage{hyperref}
\author{Construção de compiladores I}
\date{}
\title{Geração de código para L1}
\hypersetup{
 pdfauthor={Construção de compiladores I},
 pdftitle={Geração de código para L1},
 pdfkeywords={},
 pdfsubject={},
 pdfcreator={Emacs 30.1 (Org mode 9.7.11)}, 
 pdflang={English}}
\begin{document}

\maketitle
\section*{Objetivos}
\label{sec:orgbf6b58d}

\subsection*{Objetivos}
\label{sec:orge7b32b0}

\begin{itemize}
\item Apresentar a implementação de um gerador de código de máquina virtual para L1
\end{itemize}
\section*{Motivação}
\label{sec:org4cb216b}

\subsection*{Motivação}
\label{sec:org7971f03}

\begin{itemize}
\item Na aula anterior, vimos como implementar um verificador de tipos para a linguagem L3.

\item Nesta aula, vamos mostrar como implementar a geração de código para L1.
\end{itemize}
\section*{Geração de código para L1}
\label{sec:orgc36eb20}

\subsection*{Geração de código para L1}
\label{sec:org90651ee}

\begin{itemize}
\item Bastante similar a geração de código para expressões aritméticas em L0.

\item O que muda?
\begin{itemize}
\item Geração de código para atribuição.
\item Geração de código para read / print.
\end{itemize}
\end{itemize}
\subsection*{Geração de código para L1}
\label{sec:org3dd3b9f}

\begin{itemize}
\item Gerando código para atribuições
\end{itemize}

\begin{verbatim}
s1Codegen (LAssign v e1)
  = e1Codegen e1 ++ [Store v]
\end{verbatim}
\subsection*{Geração de código para L1}
\label{sec:orgddfa80f}

\begin{itemize}
\item Gerando código para read / print
\end{itemize}

\begin{verbatim}
s1Codegen (LRead s v)
  = [Push (VStr s), Print, Input, Store v]
s1Codegen (LPrint e1)
  = e1Codegen e1 ++ [Print]
\end{verbatim}
\subsection*{Geração de código para L1}
\label{sec:org9cfa781}

\begin{itemize}
\item Geração de código C
\end{itemize}

\begin{verbatim}
cL1Codegen :: L1 -> String
cL1Codegen e
  = unlines $ [ "#include <stdio.h>"
              , "// code generated for expressions"
              , "int main () {" ] ++
              (map (nest 3) (generateBody e)) ++
              [ nest 3 "return 0;"
              , "}"
              ]
    where
      nest n v = replicate n ' ' ++ v
\end{verbatim}
\subsection*{Geração de código para L1}
\label{sec:org1b3513d}

\begin{itemize}
\item Geração de código C
\end{itemize}

\begin{verbatim}
generateStmt :: S1 -> String
generateStmt (LAssign v e1)
  = unwords ["int", pretty v, "=", generateExp e1, ";"]
generateStmt (LPrint e1)
  = unwords ["printf(%d,", generateExp e1, ");"]
generateStmt (LRead s v)
  = unwords ["print(\"",s,"\");\n", "scanf(%d, &", pretty v, ")"]
\end{verbatim}
\section*{Conclusão}
\label{sec:org58e0ea0}

\subsection*{Conclusão}
\label{sec:org0dada57}

\begin{itemize}
\item Apresentamos a geração de código para a linguagem simples, L1.
\begin{itemize}
\item Máquinas virtual com memória e pilha.
\item Geração de código C.
\end{itemize}
\end{itemize}
\end{document}
