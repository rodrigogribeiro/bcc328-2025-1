% Intended LaTeX compiler: pdflatex
\documentclass[11pt]{article}
\usepackage[utf8]{inputenc}
\usepackage[T1]{fontenc}
\usepackage{graphicx}
\usepackage{longtable}
\usepackage{wrapfig}
\usepackage{rotating}
\usepackage[normalem]{ulem}
\usepackage{amsmath}
\usepackage{amssymb}
\usepackage{capt-of}
\usepackage{hyperref}
\author{Construção de compiladores I}
\date{}
\title{Análise sintática}
\hypersetup{
 pdfauthor={Construção de compiladores I},
 pdftitle={Análise sintática},
 pdfkeywords={},
 pdfsubject={},
 pdfcreator={Emacs 29.4 (Org mode 9.7.22)}, 
 pdflang={English}}
\begin{document}

\maketitle
\section*{Objetivos}
\label{sec:orgd6645e3}

\subsection*{Objetivos}
\label{sec:orge93a2d9}

\begin{itemize}
\item Apresentar a importância da etapa de análise sintática.

\item Revisar conceitos de gramáticas e linguagens livres de contexto
\end{itemize}
\subsection*{Objetivos}
\label{sec:orgcfe0114}

\begin{itemize}
\item Apresentar como representar programas como árvores de sintaxe.
\begin{itemize}
\item Como árvores de sintaxe podem ser codificadas como tipos em Haskell.
\end{itemize}

\item Apresentar a técnica de análise sintática descendente recursiva.
\end{itemize}
\section*{Análise sintática}
\label{sec:org4659ab1}

\subsection*{Análise sintática}
\label{sec:orge4886c1}

\begin{itemize}
\item Responsável por determinar se o programa atende as restrições sintáticas
da linguagem.
\end{itemize}
\subsection*{Análise sintática}
\label{sec:orgea636fd}

\begin{itemize}
\item Regras sintáticas de uma linguagem são expressas utilizando gramáticas livres de contexto.
\end{itemize}
\subsection*{Análise sintática}
\label{sec:orgcfa554c}

\begin{itemize}
\item Porque utilizar GLCs e não ERs?
\begin{itemize}
\item ERs não são capazes de representar construções simples de linguagens.
\end{itemize}
\end{itemize}
\subsection*{Análise sintática}
\label{sec:org9df55a4}

\begin{itemize}
\item Vamos considerar um fragmento de expressões formado por variáveis, constantes inteiras
adição, multiplicação.
\end{itemize}
\subsection*{Análise sintática}
\label{sec:org9d602ef}

\begin{itemize}
\item A seguinte ER representa essa linguagem:
\end{itemize}

\begin{array}{c}
base = [a..z]([a..z] | [0..9])^* \\
base((+|*)base)^*
\end{array}
\subsection*{Análise sintática}
\label{sec:orgf4bc234}

\begin{itemize}
\item A ER anterior aceita palavras como \(a * b + c\).

\item Porém, como determinar a precedência entre operadores?
\end{itemize}
\subsection*{Análise sintática}
\label{sec:org5edb33d}

\begin{itemize}
\item Podemos usar a precedência usual da aritmética.

\item Porém, não é possível impor uma ordem diferente de avaliação.
\begin{itemize}
\item Para isso, precisamos de parêntesis.
\end{itemize}
\end{itemize}
\subsection*{Análise sintática}
\label{sec:org7bc8dff}

\begin{itemize}
\item Ao incluir parêntesis, temos um problema:
\begin{itemize}
\item Como expressar usando ER que parêntesis estão corretos?
\end{itemize}
\end{itemize}
\subsection*{Análise sintática}
\label{sec:org874f3f8}

\begin{itemize}
\item Pode-se provar que a linguagem de parêntesis balanceados não é regular.
\begin{itemize}
\item Usando o lema do bombeamento.
\item Estrutura similar a \(\{0^n1^n\,|\,n\geq 0\}\).
\end{itemize}
\end{itemize}
\subsection*{Análise sintática}
\label{sec:org2557bdd}

\begin{itemize}
\item Dessa forma, precisamos utilizar GLCs para representar a estrutura sintática
de linguagens.
\end{itemize}
\subsection*{Análise sintática}
\label{sec:org8de8b58}

\begin{itemize}
\item Antes de apresentar técnicas de análise sintática, vamos revisar alguns
conceitos sobre GLCs.
\end{itemize}
\section*{Gramáticas Livres de Contexto}
\label{sec:org56fb0b2}

\subsection*{Gramáticas livres de contexto}
\label{sec:org70e583b}

\begin{itemize}
\item Uma GLC é \(G=(V,\Sigma,R,P)\), em que
\begin{itemize}
\item \(V\): conjunto de variáveis (não terminais)
\item \(\Sigma\): alfabeto (terminais)
\item \(R \subseteq V\times (V\cup\Sigma)^*\): regras (produções).
\item \(P\in V\): variável de partida.
\end{itemize}
\end{itemize}
\subsection*{Gramáticas livres de contexto}
\label{sec:org30c339f}

\begin{itemize}
\item Gramática de expressões
\end{itemize}

\begin{array}{lcl}
E & \to & (E) \,|\, E + E \,|\, E * E\,|\, num\,|\,var\\
\end{array}
\subsection*{Gramáticas livres de contexto}
\label{sec:orgefc57f9}

\begin{itemize}
\item \(V = \{E\}\)
\item \(\Sigma = \{num, var, (, ), *, +\}\)
\item \(R\): conjunto de regras da gramática.
\end{itemize}
\subsection*{Gramáticas livres de contexto}
\label{sec:org8852942}

\begin{itemize}
\item Determinamos se uma palavra pertence ou não à linguagem
de uma gramática construindo uma \textbf{derivação}
\end{itemize}
\subsection*{Gramáticas livres de contexto}
\label{sec:org0faaff3}

\begin{itemize}
\item Exemplo: Derivação de \(num + num * num\).
\end{itemize}

\begin{array}{lcl}
E       & \Rightarrow &
\end{array}
\subsection*{Gramáticas livres de contexto}
\label{sec:org70f020f}

\begin{itemize}
\item Exemplo: Derivação de \(num + num * num\).
\end{itemize}

\begin{array}{lcl}
E       & \Rightarrow & \textbf{regra } E\to E + E\\
E + E   \\
\end{array}
\subsection*{Gramáticas livres de contexto}
\label{sec:org30f5816}

\begin{itemize}
\item Exemplo: Derivação de \(num + num * num\).
\end{itemize}

\begin{array}{lcl}
E       & \Rightarrow & \textbf{regra } E\to E + E\\
E + E   & \Rightarrow & \textbf{regra } E \to num\\
\end{array}
\subsection*{Gramáticas livres de contexto}
\label{sec:org5243b11}

\begin{itemize}
\item Exemplo: Derivação de \(num + num * num\).
\end{itemize}

\begin{array}{lcl}
E       & \Rightarrow & \textbf{regra } E\to E + E\\
E + E   & \Rightarrow & \textbf{regra } E \to num\\
num + E \\
\end{array}
\subsection*{Gramáticas livres de contexto}
\label{sec:orge81636a}

\begin{itemize}
\item Exemplo: Derivação de \(num + num * num\).
\end{itemize}

\begin{array}{lcl}
E       & \Rightarrow & \textbf{regra } E\to E + E\\
E + E   & \Rightarrow & \textbf{regra } E \to num\\
num + E & \Rightarrow & \textbf{regra } E \to E * E\\
num + E * E\\
\end{array}
\subsection*{Gramáticas livres de contexto}
\label{sec:org30acd2f}

\begin{itemize}
\item Exemplo: Derivação de \(num + num * num\).
\end{itemize}

\begin{array}{lcl}
E       & \Rightarrow & \textbf{regra } E\to E + E\\
E + E   & \Rightarrow & \textbf{regra } E \to num\\
num + E & \Rightarrow & \textbf{regra } E \to E * E\\
num + E * E & \Rightarrow & \textbf{regra } E \to num\\
\end{array}
\subsection*{Gramáticas livres de contexto}
\label{sec:org162e271}

\begin{itemize}
\item Exemplo: Derivação de \(num + num * num\).
\end{itemize}

\begin{array}{lcl}
E       & \Rightarrow & \textbf{regra } E\to E + E\\
E + E   & \Rightarrow & \textbf{regra } E \to num\\
num + E & \Rightarrow & \textbf{regra } E \to E * E\\
num + E * E & \Rightarrow & \textbf{regra } E \to num\\
num + num * E \\
\end{array}
\subsection*{Gramáticas livres de contexto}
\label{sec:org54436f6}

\begin{itemize}
\item Exemplo: Derivação de \(num + num * num\).
\end{itemize}

\begin{array}{lcl}
E       & \Rightarrow & \textbf{regra } E\to E + E\\
E + E   & \Rightarrow & \textbf{regra } E \to num\\
num + E & \Rightarrow & \textbf{regra } E \to E * E\\
num + E * E & \Rightarrow & \textbf{regra } E \to num\\
num + num * E & \Rightarrow & \textbf{regra } E \to num \\
num + num * num
\end{array}
\subsection*{Gramáticas livres de contexto}
\label{sec:org0e823fa}

\begin{itemize}
\item O exemplo anterior foi de uma \textbf{derivação mais à esquerda}
\begin{itemize}
\item Expande-se o não terminal mais a esquerda.
\end{itemize}
\end{itemize}
\subsection*{Gramáticas livres de contexto}
\label{sec:orgefe8b3e}

\begin{itemize}
\item Note que essa gramática de expressões permite:
\end{itemize}

\begin{array}{lcl}
E       & \Rightarrow & \textbf{regra } E\to E * E\\
E * E   \\
\end{array}
\subsection*{Gramáticas livres de contexto}
\label{sec:org078113e}

\begin{itemize}
\item Com isso temos \textbf{duas} derivações distintas para a mesma palavra.

\item Isso torna a gramática de exemplo \textbf{ambígua}.
\end{itemize}
\subsection*{Gramáticas livres de contexto}
\label{sec:orgecf2ed3}

\begin{itemize}
\item Em algumas situações é necessário modificar regras de uma gramática para usar certas técnicas de análise sintática.

\item Veremos algumas dessas técnicas.
\end{itemize}
\section*{Transformações de gramáticas}
\label{sec:org6c3b481}

\subsection*{Transformações de gramáticas}
\label{sec:org622a21c}

\begin{itemize}
\item Fatoração à esquerda: Evitar mais de uma regra com o mesmo prefixo
\end{itemize}
\subsection*{Transformações de gramáticas}
\label{sec:org44e9e26}

\begin{itemize}
\item Exemplo:
\end{itemize}

\begin{array}{lcl}
  A & \to & xz \,|\, xy\,|\,v
\end{array}

\begin{itemize}
\item pode ser transformada em:
\end{itemize}

\begin{array}{lcl}
  A & \to & xZ\,|\,v\\
  Z & \to & z \,|\,y
\end{array}
\subsection*{Transformações de gramáticas}
\label{sec:org6eea7a3}

\begin{itemize}
\item Introdução de prioridades.
\begin{itemize}
\item Problema comum em linguagens de programação com operadores.
\item Impor ordem de precedência na ausência de parêntesis.
\end{itemize}
\end{itemize}
\subsection*{Transformações de gramáticas}
\label{sec:org6ba1bfe}

\begin{itemize}
\item Forma geral para introduzir prioridades:
\begin{itemize}
\item \(E_i\): expressões com precedência de nível \(i\).
\item Maior precedência: mais profundo.
\end{itemize}
\end{itemize}

\begin{array}{lcl}
E_i & \to & E_{i + 1} \,|\, E_i Op_i E_{i + 1}
\end{array}
\subsection*{Transformação de gramáticas}
\label{sec:org97d2b75}

\begin{itemize}
\item Exemplo:
\begin{itemize}
\item Multiplicação com predência maior que adição.
\end{itemize}
\end{itemize}

\begin{array}{lcl}
E & \to & n \,|\,E + E\,|\,E * E\\
\end{array}
\subsection*{Transformação de gramáticas}
\label{sec:orgb0d561a}

\begin{itemize}
\item Exemplo
\end{itemize}

\begin{array}{lcl}
E_1 & \to & E_1 + E_2\,|\,E_2 \\
E_2 & \to & E_2 * E_3\,|\,E_3 \\
E_3 & \to & n\\
\end{array}
\subsection*{Transformações de gramáticas}
\label{sec:org53aa61b}

\begin{itemize}
\item Eliminar recursão à esquerda
\begin{itemize}
\item Transformar em recursão à direita.
\end{itemize}
\end{itemize}

\begin{array}{lcl}
A & \to & Ay_1\,|\,...\,|\,Ay_n\,|\,w_1\,|\,...\,|\,w_k\\
\end{array}
\subsection*{Transformações de gramáticas}
\label{sec:orgc6559dd}

\begin{itemize}
\item Pode ser transformada em
\end{itemize}

\begin{array}{lcl}
A & \to & w_1Z\,|\,...\,|\,w_kZ\,|\,w_1\,...\,|\,w_k\\
Z & \to & y_1Z\,|\,...\,|\,y_nZ\,|\,y_1\,...\,|\,y_n\\
\end{array}
\subsection*{Transformação de gramáticas}
\label{sec:org06da2b6}

\begin{itemize}
\item Eliminar recursão a esquerda.
\begin{itemize}
\item Resolução no quadro
\end{itemize}
\end{itemize}

\begin{array}{lcl}
   S & \to & Aa\,|\,b\\
   A & \to & Ac\,|\,Sd\,|\,\lambda\\
\end{array}
\section*{Árvores de sintaxe}
\label{sec:org3f83ce8}

\subsection*{Árvores de sintaxe}
\label{sec:orgb8bdec4}

\begin{itemize}
\item Em teoria de linguagens, representamos derivações de uma gramática por \textbf{\textbf{árvores de derivação}}.

\item Uma árvore de sintaxe deve representar a estrutura da derivação de um programa.
\end{itemize}
\subsection*{Árvores de sintaxe}
\label{sec:orgda63e33}

\begin{itemize}
\item Estratégia para definir árvores de sintaxe
\begin{itemize}
\item Um tipo para cada não terminal da gramática.
\item Cada regra de um não terminal, é um construtor do tipo.
\end{itemize}
\end{itemize}
\subsection*{Árvores de sintaxe}
\label{sec:orgdece0ed}

\begin{itemize}
\item Qual a estrutura da árvore de sintaxe:
\end{itemize}

\begin{array}{lcl}
E & \to & num \,|\,var\,|\,(E)\,|\,E+E\,|\,E * E\\
\end{array}
\subsection*{Árvores de sintaxe}
\label{sec:org73814e9}

\begin{itemize}
\item Árvore de sintaxe
\end{itemize}

\begin{verbatim}
data Exp = Const Int
         | Var String
         | Add Exp Exp
         | Mul Exp Exp
\end{verbatim}
\subsection*{Árvores de sintaxe}
\label{sec:orga583c67}

\begin{itemize}
\item Porque não uma construção para parêntesis?
\begin{itemize}
\item São usados apenas para determinar precedência
\item A rigor, parêntesis não tem significado após análise sintática.
\end{itemize}
\end{itemize}
\subsection*{Árvores de sintaxe}
\label{sec:orgebedecb}

\begin{itemize}
\item O tipo anterior é um exemplo de sintaxe \textbf{\textbf{abstrata}}
\begin{itemize}
\item Elimina detalhes irrelevantes para o significado do programa.
\end{itemize}
\item Código escrito do programa usa a sintaxe \textbf{\textbf{concreta}}.
\end{itemize}
\subsection*{Árvores de sintaxe}
\label{sec:org26be394}

\begin{itemize}
\item Considere a seguinte gramática:
\end{itemize}

\begin{array}{lcl}
S & \to & S\:S\:|\:s
\end{array}
\subsection*{Árvores de sintaxe}
\label{sec:org309e3d7}

\begin{itemize}
\item Representando a árvore de sintaxe
\end{itemize}

\begin{verbatim}
data S = Rule1 S S | Rule2 Char
\end{verbatim}
\subsection*{Árvores de sintaxe}
\label{sec:orgd522cc4}

\begin{itemize}
\item Considere a tarefa de produzir a string representada pela árvore
\end{itemize}

\begin{verbatim}
pprS :: S -> String
pprS (Rule1 s1 s2) = pprS s1 ++ pprS s2
pprS (Rule2 _) = "s"
\end{verbatim}
\subsection*{Árvores de sintaxe}
\label{sec:org226edba}

\begin{itemize}
\item Note que o construtor \texttt{Rule2 Char} não usa o caracter que armazena
\begin{itemize}
\item Sempre vamos produzir o caractere \texttt{s}.
\end{itemize}
\end{itemize}
\subsection*{Árvores de sintaxe}
\label{sec:org306864d}

\begin{itemize}
\item Podemos refinar a árvore para
\end{itemize}

\begin{verbatim}
data SA = Rule1 SA SA | Rule2
\end{verbatim}
\subsection*{Árvores de sintaxe}
\label{sec:orge5d9b8a}

\begin{itemize}
\item Refinando a função de impressão
\end{itemize}

\begin{verbatim}
pprS :: SA -> String
pprS (Rule1 s1 s2) = pprS s1 ++ pprS s2
pprS Rule2 = "s"
\end{verbatim}
\section*{Análise descendente}
\label{sec:orga40909b}

\subsection*{Análise descendente}
\label{sec:orgbd70ae8}

\begin{itemize}
\item Na apresentação do compilador de expressões, implementamos funções simples para um analisador descendente

\item Apesar de possuir uma implementação simples:
\begin{itemize}
\item Não é eficiente
\item Não permite uma maneira adequada para lidar com erros de análise sintática.
\end{itemize}
\end{itemize}
\subsection*{Análise descendente}
\label{sec:org245683c}

\begin{itemize}
\item Vamos utilizar a biblioteca \texttt{megaparsec}
\begin{itemize}
\item Permite a construção de analisadores descendentes eficientes.
\item Bom suporte a mensagens de erro.
\end{itemize}
\end{itemize}
\subsection*{Análise descendente}
\label{sec:org782ed52}

\begin{itemize}
\item Excelente documentação disponível on-line:
\end{itemize}

\url{https://markkarpov.com/tutorial/megaparsec.html}
\subsection*{Análise descendente}
\label{sec:orgeb89e10}

\begin{itemize}
\item Vamos apresentar a implementação do parser de expressões usando \texttt{megaparsec}

\item Exemplo disponível no módulo \texttt{Megaparsec.ParserExample} do repositório.
\end{itemize}
\subsection*{Análise descendente}
\label{sec:org5035ef2}

\begin{itemize}
\item Primeiro passo: definir um tipo para os parsers e erros
\end{itemize}

\begin{verbatim}
type Parser = Parsec Void String

type ParserError = ParseErrorBundle String Void
\end{verbatim}
\subsection*{Análise descendente}
\label{sec:orgdfb0065}

\begin{itemize}
\item Segundo passo: definir um analisador léxico.
\end{itemize}

\begin{verbatim}
slexer :: Parser ()
slexer = L.space space1
                 (L.skipLineComment "//")
                 (L.skipBlockComment "/*" "*/")
\end{verbatim}
\subsection*{Análise descendente}
\label{sec:org5da10db}

\begin{itemize}
\item Definindo funções simples.
\end{itemize}

\begin{verbatim}
symbol :: String -> Parser String
symbol s = L.symbol slexer s
\end{verbatim}
\subsection*{Análise descendente}
\label{sec:org0721a7a}

\begin{itemize}
\item Lidando com parêntesis
\end{itemize}

\begin{verbatim}
parens :: Parser a -> Parser a
parens = between (symbol "(") (symbol ")")
\end{verbatim}
\subsection*{Análise descendente}
\label{sec:org444cd86}

\begin{itemize}
\item Adicionando a capacidade de eliminar espaços e comentários em um parser qualquer.
\end{itemize}

\begin{verbatim}
lexeme :: Parser a -> Parser a
lexeme = L.lexeme slexer
\end{verbatim}
\subsection*{Análise descendente}
\label{sec:org792f05d}

\begin{itemize}
\item Processando números
\end{itemize}

\begin{verbatim}
integer :: Parser Int
integer = lexeme L.decimal
\end{verbatim}
\subsection*{Análise descendente}
\label{sec:orga99f5ce}

\begin{itemize}
\item Processando um fator
\end{itemize}

\begin{verbatim}
pFactor :: Parser Exp
pFactor = choice [ Const <$> integer
                 , parens pExp
                 ]

\end{verbatim}
\subsection*{Análise descendente}
\label{sec:org145acc4}

\begin{itemize}
\item Para criar o parser de expressões, vamos usar a função \texttt{makeExprParser} que constrói o parser a partir de uma tabela de precedências.
\end{itemize}
\subsection*{Análise descendente}
\label{sec:org6d5264b}

\begin{itemize}
\item Definindo uma função para criar a precedência de um operador binário.
\begin{itemize}
\item Pode-se definir operadores unários pré-fixados (\texttt{Prefix}) e pós-fixados (\texttt{Postfix})
\end{itemize}
\end{itemize}

\begin{verbatim}
binary :: String -> (Exp -> Exp -> Exp) -> Operator Parser Exp
binary name f = InfixL (f <$ symbol name)
\end{verbatim}
\subsection*{Análise descendente}
\label{sec:orgdd8c884}

\begin{itemize}
\item Usando a função anterior, podemos criar a tabela de precedências.
\begin{itemize}
\item Maiores precedências aparecem primeiro na tabela.
\end{itemize}
\end{itemize}

\begin{verbatim}
optable :: [[Operator Parser Exp]]
optable = [
            [binary "*" Mul]
          , [binary "+" Add]
          ]
\end{verbatim}
\subsection*{Análise descendente}
\label{sec:orgcad6fbe}

\begin{itemize}
\item Parser de expressões
\end{itemize}

\begin{verbatim}
pExp :: Parser Exp
pExp = makeExprParser pFactor optable
\end{verbatim}
\subsection*{Análise descendente}
\label{sec:org5c3ca01}

\begin{itemize}
\item Podemos processar qualquer gramática usando analisadores descendentes?
\begin{itemize}
\item Não: essa técnica aplica-se a gramáticas da classe LL(k).
\end{itemize}
\end{itemize}
\subsection*{Análise descendente}
\label{sec:org95202c6}

\begin{itemize}
\item Gramáticas LL(k)
\begin{itemize}
\item \textbf{\textbf{L}} : Entrada processada da esquerda para a direita
\item \textbf{\textbf{L}}: Produzindo uma derivação mais a esquerda
\item \textbf{\textbf{k}}: tomando a decisão usando até \textbf{\textbf{k}} tokens da entrada.
\end{itemize}
\end{itemize}
\subsection*{Análise descendente}
\label{sec:orgb4e428d}

\begin{itemize}
\item Gramáticas LL(k)
\begin{itemize}
\item Não possuem recursão à esquerda
\item Não possuem fatores comuns à esquerda
\end{itemize}
\item De maneira geral, gramáticas LL(k) não possuem \textbf{\textbf{ambiguidade}}
\end{itemize}
\subsection*{Análise descendente}
\label{sec:org95324ed}

\begin{itemize}
\item Então, para determinar se uma gramática é LL(k), basta determinar se ela é ou não ámbigua\ldots{}
\end{itemize}
\subsection*{Análise descendente}
\label{sec:org75d977c}

\begin{itemize}
\item Ambiguidade de gramáticas livres de contexto é um problema indecidível, no caso geral.
\begin{itemize}
\item Pode-se reduzir o problema de correspondência de Post a ele.
\end{itemize}
\end{itemize}
\subsection*{Análise descendente}
\label{sec:org8c60347}

\begin{itemize}
\item Vantagens:
\begin{itemize}
\item Analisadores descendentes são eficientes, para \(k = 1\).
\item Simples implementação.
\end{itemize}
\end{itemize}
\subsection*{Análise descendente}
\label{sec:orge388de2}

\begin{itemize}
\item Desvantagens:
\begin{itemize}
\item Não são capazes de lidar com gramáticas com regras recursivas à esquerda.
\item Regras não devem possuir fatores comuns à esquerda.
\end{itemize}
\end{itemize}
\subsection*{Análise descendente}
\label{sec:org22484bb}

\begin{itemize}
\item Algum compilador usa essa técnica?
\begin{itemize}
\item Analisador sintático de Lua e Go é descendente recursivo.
\item Analisador sintático de Clang é baseado nesta técnica.
\end{itemize}
\end{itemize}
\section*{Conclusão}
\label{sec:orgc3a6306}

\subsection*{Conclusão}
\label{sec:org0692806}

\begin{itemize}
\item Nesta aula:
\begin{itemize}
\item Importântica da análise sintática em um compilador.
\item Revisamos conceitos de gramáticas livres de contexto e transformações sobre estas.
\end{itemize}
\end{itemize}
\subsection*{Conclusão}
\label{sec:org5eb49da}

\begin{itemize}
\item Nesta aula:
\begin{itemize}
\item Discutimos sobre sintaxe concreta e abstrata.
\item Mostramos como deduzir uma árvore de sintaxe a partir de uma gramática.
\end{itemize}
\end{itemize}
\subsection*{Conclusão}
\label{sec:org6f6d473}

\begin{itemize}
\item Nesta aula:
\begin{itemize}
\item Apresentamos a técnica de análise descendente recursiva.
\item Usamos a biblioteca \texttt{megaparsec} para construção de um analisador descendente.
\item Discutimos vantagens e desvantagens de analisadores descendentes.
\end{itemize}
\end{itemize}
\subsection*{Conclusão}
\label{sec:orgd7cf339}

\begin{itemize}
\item Próxima aula:

\begin{itemize}
\item Análise sintática preditiva LL(1).
\end{itemize}
\end{itemize}
\end{document}
